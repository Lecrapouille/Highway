\documentclass[french, a4paper, 10pt]{article}
\usepackage[french]{babel}
%\usepackage[latin1]{inputenc}
\usepackage[utf8]{inputenc}
\usepackage{amstext}
\usepackage{amsmath}
\usepackage{graphicx}
\usepackage{longtable}
\usepackage{rotating}
\usepackage{fancyhdr}
\usepackage{amssymb}
\usepackage{picins}
\usepackage{multicol}
\usepackage{footnote}
%\usepackage{xy}
%\xyoption{all}

%%
\setlength{\textwidth}{18cm}
\setlength{\hoffset}{-3cm}
\setlength{\voffset}{-4cm}
\setlength{\textheight}{28cm}
%%

\newcommand{\dessin}[4]{
\begin{figure}[htb]
\centering
\includegraphics[scale= #2]{#1}
\caption{#3}
\label{#4}
\end{figure}}
%%

\newcommand{\yinifnt}{\fontencoding{OT1}\fontfamily{yinit}\fontsize{40}{60}\selectfont}
\newtheorem{remarque}{Remarque} \newcommand{\AAA}{{\mathcal A}}
\newcommand{\BB}{{\mathcal B}} \newcommand{\CC}{{\mathcal C}}
\newcommand{\DD}{{\mathcal D}} \newcommand{\EE}{{\mathcal E}}
\newcommand{\FF}{{\mathcal F}} \newcommand{\GG}{{\mathcal G}}
\newcommand{\HH}{{\mathcal H}} \newcommand{\II}{{\mathcal I}}
\newcommand{\JJ}{{\mathcal J}} \newcommand{\KK}{{\mathcal K}}
\newcommand{\LL}{{\mathcal L}} \newcommand{\MM}{{\mathcal M}}
\newcommand{\NN}{{\mathcal N}} \newcommand{\OO}{{\mathcal O}}
\newcommand{\PP}{{\mathcal P}} \newcommand{\QQ}{{\mathcal Q}}
\newcommand{\RR}{{\mathcal R}} \newcommand{\SSS}{{\mathcal S}}
\newcommand{\TT}{{\mathcal T}} \newcommand{\UU}{{\mathcal U}}
\newcommand{\VV}{{\mathcal V}} \newcommand{\WW}{{\mathcal W}}
\newcommand{\XX}{{\mathcal X}} \newcommand{\ZZ}{{\mathcal Z}}
\newcommand{\bbR}{{\mathbb R}} \newcommand{\bbD}{{\mathbb D}}
\newcommand{\bbO}{{\mathbb O}} \newcommand{\bbS}{{\mathbb S}}
\newcommand{\bbE}{{\mathbb E}} \newcommand{\bbN}{{\mathbb N}}
\newcommand{\bbM}{{\mathbb M}} \newcommand{\bbV}{{\mathbb V}}
\newcommand{\bbC}{{\mathbb K}} \newcommand{\bbF}{{\mathbb F}}
\newcommand{\bbP}{{\mathbb P}}
\newcommand{\www}{{\mathfrak w \;}} \newcommand{\fff}{{\mathfrak f \;}}
\newcommand{\nnn}{{\mathfrak n \;}} \newcommand{\aaa}{{\mathfrak a \;}}
\newcommand{\hhh}{{\mathfrak h \;}}
%%

\title{Projet d'innovation: Réconcilier connectivisme et symbolisme appliqué à une
  voiture autonome évoluant dans un parking pour se garer}

\author{QUADRAT Quentin}

\begin{document}
\maketitle
%\tableofcontents
%\newpage

Y. Le Cun (prix Turing) dans son livre \emph{Quand la machine apprend} (édition
Odile Jacob 2019) à dit : \emph{Qu'est-ce qui permet alors à la plupart des
humains d'apprendre à conduire en une vingtaine d'heures de pratique et très peu
de supervision, sans causer ensuite d'accidents (pour la plupart d'entre nous) ?
L'apprentissage supervisé et l'apprentissage par renforcement n'y suffisent
pas. Il reste à inventer un nouveau paradigme pour que la machine puisse égaler
l'apprentissage humain ou animal.}

Le travail de J. C. Eccles (prix nobel de médecine) dans son livre
\emph{Evolution du cerveau et creation de la conscience}, édition Flammarion
1989, nous fait penser que le mot \emph{animal} dans la phrase de Le Cun montre
qu'il n'aurait pas compris que ce \emph{nouveau paradigme} pourrait simplement
être la logique et le langage. En effet, Eccles montre que la partie droite du
cerveau des humains serait liée à la reconnaissance de forme et à la
spatialization des objets alors que la partie gauche serait liée à la logique et
au langage. En suivant son raisonement on pourrait faire l'analogie entre la
partie droite du cerveau à l'IA actuelle (reconnaissance de forme,
spatialisation en utilisant des outils tels que le calcul numérique, gradient
stochastique, traitement du signal ...), domaine nommé connectivisme mais que
l'on nommera dans ce document d'IA droite. La partie gauche du cerveau se
rapprocherait de l'IA années 80 (systèmes experts étudiant le langage naturel
avec des outils mathématique tel que la logique et des langages de programmation
comme Prolog et Lisp) et domaine nommé symbolisme mais que l'on nommera dans ce
document d'IA gauche.

Il ne nous semble pas qu'il y ait eu de réelles tentatives de réunir IA droite
et IA gauche.

\section{Contexte historique du connectivisme et du logique}

Les tentatives de modélisation de ce que serait l'intelligence artificielle date
des années 1940.  Le graphique suivant~\ref{ConnectionistSymbolic}, provenant de
la vidéo de Jean-Luc Parouty intitulée \emph{La controverse des neurones} de sa
chaîne Youtube \emph{CNRS - Formation FIDLE}, montre l'évolution de l'influence
académique entre les connectionistes (réseaux de neuronnes) et symbolique
(logique). Il représente le nombre de publication dans l'un de ces deux domaines
au cours des années. En 1940 apparaît le premier perceptron linéaire pour
classifier les différentes espèces d'Iris, mais par la suite il n'y a eu que peu
de progrès du au fait que les classifieurs non linéaires n'avaient pas été
encore inventés (il faudra attendre 1986). En 1969 marque le début du mouvement
symbolique grâce à l'emmergence des ordinateurs et à la création de systèmes
experts. En conséquence, la logique a remplacé le connexionnisme mais lui aussi
se heurte a aucun progrès majeur. S'en suit l'annulation et le gel des projets
de recherche en intelligence artificiel jusqu'à deux innovations majeures en
1986 et 1989 qui ont réussité les réseaux de neurones: d'abord la
rétro-propagation par Rumelhart puis l'invention des réseaux de convolution par
Y. Le Cun. La puissance de calcul des puces modernes ont permis la renaissance
et l'explosition en popularité du connectivisme.

\dessin{pics/ConnectionistSymbolic}{0.40}{Evolution de
l'influence académique entre IA droite et IA gauche.}{ConnectionistSymbolic}

\section{Vitesse d'apprentissage entre les singes, les ordinateurs et les humains}

\subsection{Vitesse d'apprentissage entre les singes et les humains}

J. C. Eccles dans son livre \emph{Evolution du cerveau et creation de la
conscience}, édition Flammarion 1989, page 222 paragraphe 7.5.3 dit \emph{qu'il
est étonnant qu'il faille jusqu'à 4000 simulations -- 1000 par semaine --, même
chez un singe intelligent pour qu'il apprenne à associer un signal lumineux à un
mouvement aussi simple que lever un levier. En effet un jeune enfant (d'âge
préscolaire) y parvient au bout de quelques essais ... Les chimpanzés sont très
gênés face à une situation nouvelle et se montrent incapables de traiter par le
langage une difficulté. L'évolution des hominiens met en relief le rôle énorme
du langage dans le processus d'évolution. }

\subsection{Evolution du cerveau et développement du langage}

L'évolution du cerveau et du développement du langage chez les humains serait
arrivée tardivement dans l'évolution:

J. C. Eccles dans son livre \emph{Evolution du cerveau et creation de la
conscience}, page 121 paragraphe 4.7: \emph{Au cours de l'évolution qui va du
singe au primate supérieur oui à l'homme aucun changement n'a été relevé au
niveau des voies auditives jusqu'aux aires auditives primaires et secondaires du
cortex. Mais ensuite, les différences sont prodigieuses. Rien chez les primates
supérieures ne correspond à l'aire antérieure du langage découvert par
Broca. Plus étonnant encore, c'est à peine si l'on détecte dans le cerveau de
l'orang outan et du gibbon les aires pariétales inférieures et les gyrins
angulaires et supra-marginal, si vastes chez l'homme.}

Eccles, page 278 paragraphe 9.6, explique la différence entre

\emph{L'hémisphère gauche effectue une synthèse dans l'espace, l'hémisphère
gauche une analyse dans le temps. Il manque à l'hémisphère droit un analyseur du
langage et à l'hémisphère gauche un synthétiseur de forme. Le grand succès de
l'évolution des hominidés a été assuré par l'organisation asymétrique qui a
potentiellement doublé la capacité du cortex. Ce qui a permis un grand
accroissement du net-neocortex sans trop d'augmentation des risques dus à la
naissance d'enfants à la tête de plus en plus grosse.}

Toujours selon les dires d'Eccles

\begin{savenotes}
\begin{center}
  \begin{tabular}{|c|c|} \hline
    Hémisphere dominant (gauche) & Hémisphere mineur (droit) \\ \hline
    Conscience individuelle & Conscience d'exister \\ \hline
    Verbal  & Presque non verbal \\ \hline
    Designation linguistique   & Musical \\ \hline
    Ideation   & Sens de l'image et de la forme \\ \hline
    Ressemblance conceptuelle  & Ressemblance visuelle \\ \hline
    Analyse à long terme & Holiste\footnote{la tendance dans la nature à constituer des ensembles qui sont supérieurs à la somme de leurs parties, au travers de l'évolution créatrice} : Images \\ \hline
    Opération arithmétique et logique & Opération spatiale et géométrique \\ \hline
  \end{tabular}
\end{center}
\end{savenotes}

\subsection{Différence d'apprentissage entre machine et humain}

\section{Poposition d'amelioration}

IA droite, IA actuelle (reconnaissance de forme : spatiale) outils : numérique,
gradient stochastique, traitement du signal.  IA gauche, IA années 80 (systèmes
experts : langage) outils : logique, langage (prolog, lisp).

\section{IA droite: Techniques de spatialization}

\subsection{Segmentation d'images}

L'image suivante provient du simulateur de voiture Carla qui permet de simuler différents
capteurs de la voiture dont des caméras mais aussi simuler le résultat de réseau de convolution
de segmentation d'image. La segmentation permet d'obtenir une image en fausse couleur où chaque
élément est décrit par une couleur servant d'identifiant unique (par exemple rouge pour les piétions,
jaune pour les panneaux de signalisation ...).

A partir d'une image simple il est difficile d'estimer la profondeur, un capteur de type Lidar peut être
utilisé mais il reste un capteur volumineux et cher. Une alternative possible est la stéréo-vision.

https://github.com/DidierRLopes/SelfDrivingCars/blob/master/3.%20Visual%20Perception%20for%20Self-Driving%20Cars/1.%20Basics%20of%203D%20Computer%20Vision/z_assignment/Applyin%20Stereo%20Depth%20to%20a%20Driving%20Scenario%20(solution).ipynb

\dessin{pics/segmentation}{0.40}{Segmentation}{segmentation}


\dessin{pics/captioning}{0.40}{Captioning}{captioning}

\subsection{State of art for Plannification}

Current way for solving the problematic of self-driving cars an be decomposed
into hierarchy of sub problems in which each sub problem has its own objective
function and constraints to be solved.

\begin{itemize}
\item[$\bullet$] Mission planner
\item[$\bullet$] Behavioral planner
\item[$\bullet$] Local Planner
\item[$\bullet$] Vehicle Control
\end{itemize}

Mission planner: Since distances are large (> 100 m to kilometers) we often
ignore aspects of motion planning problems such as obstacles, regulatory
elements but focus on road connections. Instead, highest level mainly focus on
map-level navigation are used: from vehicle position to a a given destination,
and therefore mainly based on graph theory (roads and intersection), Dijkstra
algorithm (for finding short path).

\dessin{pics/map}{0.40}{}{map}

Behavioral planner: will focus on driving scenario and decide which maneuver to
execute based on other agents in the workspace and rules of the road, recognize
which maneuvers are safe to make in a given driving scenario (ie turn at in
intersection). There is currently no straight algorithm solving this
problem. This is mainly based on hierarchy state-machine but since
state-machines quickly grows, some other methods are also used: such as –
machine-learning or – rule-based languages (PDDL, Prolog ...). The following
figure depicts how kind of rule-based languages are based on:

Local Planner: based on kinematics equations will focus on generating feasible,
collision-free paths and comfortable velocity profiles. Then decomposed into
path planning and profile generation mainly based on trajectory optimization
(with objective and cost functions such as minimizing the error on trajectory
deviation ...)

Vehicle Control: is purely law control equations such as PID, MPC
controllers. They take as input the desired vehicle speed and steering apply
values to actuators and from sensors observe their own states and minimize the
error between references and observation.


Seul Behavioral planner n'est pas de solution toute faite.

Behavioral planner a besoin de connaitre la carte. Avec la logique pur, la
voiture en lisant les panneaux pourrait découvrir le bon chemin a parcourir.

\subsection{Plannification}

\dessin{pics/autoroute}{0.40}{}{autoroute}

Some Prolog clauses (in human language) could be:
\begin{itemize}
\item[$\bullet$] I'm on the highway
\item[$\bullet$] I'm in the middle lane.
\item[$\bullet$] I see a white car in front of me.
\item[$\bullet$] I see no car on the left lane
\item[$\bullet$] The highway has 3 lanes
\item[$\bullet$] There is a white continuous lane on the right
\item[$\bullet$] To go to "Dan Ryan Expy" is on the 1st lane
\end{itemize}

Some interrogation on the Prolog database:
\begin{itemize}
\item[$\bullet$] Is there a car in front of me ? yes!
\item[$\bullet$] To which maximal speed can I go ? 120 km/h (since we are on a highway)
\item[$\bullet$] Can I go to 120 km/h now ? no (because of the presence of the previous car).
\item[$\bullet$] Can I turn to the right ? no (because of the continuous white lane)
\item[$\bullet$] Can I turn to the left ? yes (because there is no car on the left)
\item[$\bullet$] Can I go to "Dan Ryan Expy" ? no (because  of the continuous white lane)
\end{itemize}

Logic can be converted to law control regulation. For example:
\begin{itemize}
\item[$\bullet$] Can I turn to the left ? yes
\item[$\bullet$] Turn to the left! Activation of the forward path pursuit.
\end{itemize}

\subsection{Langages de logique}

Prolog

\dessin{pics/parking}{0.40}{}{parking}

\begin{verbatim}
spot(p1).
spot(p2).
spot(p3).
spot(p4).

parked(v1,p1).
parked(v2,p2).
parked(v3,p4).

empty(X) :- spot(X), not(parked(X,Y))
\end{verbatim}

PDDL

\end{document}
